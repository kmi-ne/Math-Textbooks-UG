%%%%% 条件分岐
\newcommand{\myboolTF}[3]{\csname if#1\endcsname #2 \else #3 \fi}
\newcommand{\myboolT}[2]{\csname if#1\endcsname #2 \else \fi}
\newcommand{\myboolF}[2]{\csname if#1\endcsname \else #2 \fi}

% 
% 
% 

%%%%% ページレイアウト
\myboolTF{BoolUseDraft}{
  \setpagelayout+{vmargin=8truemm, inner=40truemm, outer=10truemm, includemp, nomarginpar}
}{
  \setpagelayout+{vmargin=8truemm, hmargin=10truemm, includemp, nomarginpar}
}

% 
% 
% 

%%%%% フォント/書体
\ifxetex
  \setCJKsansfont[BoldFont=HaranoAjiGothic-Bold]{HaranoAjiGothic-Medium}
  \setCJKmonofont[BoldFont=HaranoAjiGothic-Bold]{HaranoAjiGothic-Medium}
\else\fi
\ifluatex
  \ltjsetparameter{jacharrange={-2,-3}}
  \setsansjfont{HaranoAjiGothic}[
    UprightFont={*-Medium},
    BoldFont={*-Bold},
  ]
\else\fi
\PassOptionsToPackage{math-style=ISO, bold-style=upright}{unicode-math}
\usepackage[olddefault, newcmbb, varnothing]{fontsetup}
\setmathfont[range={scr, bfscr}, StylisticSet=1]{NewCMMath-Regular}
\DeclareMathAlphabet{\cl}{OMS}{cmsy}{m}{n} % 旧チャンセリー筆記体
\setsansfont{nimbussans}
\usepackage{manfnt}
\usepackage[old]{old-arrows}
\usepackage{leftidx}
\ifluatex
  \usepackage{longmath}
\else\fi

% 
% 
% 

%%%%% 色
\usepackage{xcolor}
\definecolor{mygreen}{HTML}{70b548}
\definecolor{myblue}{HTML}{6e9ec2}
\definecolor{myred}{HTML}{c26e7c}
\definecolor{myyellow}{HTML}{b8ba05}
\definecolor{mygray}{HTML}{878787}
\definecolor{mywhitegray}{HTML}{cccccc}
\definecolor{myurlcolor}{HTML}{ad0d00}
\definecolor{mylinkcolor}{HTML}{2a4f99}
\definecolor{mycitecolor}{HTML}{2a4f99}
\newcommand{\gx}[1]{\textcolor{mygreen}{#1}}
\newcommand{\bx}[1]{\textcolor{myblue}{#1}}
\newcommand{\rx}[1]{\textcolor{myred}{#1}}
\newcommand{\Gx}[1]{\textcolor{mygray}{#1}}

% 
% 
% 

%%%%% 目次
\setcounter{tocdepth}{4}

% 
% 
% 

%%%%% ヘッダー/フッター/見出し
\renewcommand{\plainifnotempty}{} % クラスファイルによるページスタイル強制変更を除去
\usepackage[pagestyles, clearempty, toctitles, explicit]{titlesec}
\newcommand{\headsize}{\small}

\makeatletter
\newcommand{\labelchapter}{\@chapapp \thechapter \@chappos}
\newcommand{\labelsection}{\@secapp \thesection \@secpos}
\makeatother

%%% ヘッダー/フッター
\newpagestyle{nc-ns}[\headsize\gs]{%
  \headrule
  \sethead[\textit{\thepage}][\labelchapter\quad\chaptertitle][]% 偶数ページ
  {}{\labelsection\quad\sectiontitle}{\textit{\thepage}}% 奇数ページ
}
\newpagestyle{nc-s}[\headsize\gs]{%
  \headrule
  \sethead[\textit{\thepage}][\labelchapter\quad\chaptertitle][]% 偶数ページ
  {}{\sectiontitle}{\textit{\thepage}}% 奇数ページ
}
\newpagestyle{nc-nc}[\headsize\gs]{%
  \headrule
  \sethead[\textit{\thepage}][\labelchapter\quad\chaptertitle][]% 偶数ページ
  {}{\labelchapter\quad\chaptertitle}{\textit{\thepage}}% 奇数ページ
}
\newpagestyle{c-c}[\headsize\gs]{%
  \headrule
  \sethead[\textit{\thepage}][\chaptertitle][]% 偶数ページ
  {}{\chaptertitle}{\textit{\thepage}}% 奇数ページ
}

%%% 見出し
\titleformat{\chapter}[display]{\thispagestyle{empty}\filleft\gsb}{\fontsize{80}{0}\selectfont\thechapter}{20pt}{\huge #1}
\titlespacing*{\chapter}{0pt}{50pt}{40pt}

\titleformat{\section}[display]{\gs}{\large\labelsection\hspace{.5em}\titlerule}{3pt}{\Large #1}
\titleformat{name=\section,numberless}{\gs\Large}{}{0em}{#1\hspace{.5em}\titlerule}
\titlespacing*{\section}{0pt}{15pt}{10pt}
\titlespacing*{name=\section,numberless}{0pt}{15pt}{10pt}

\titleformat{\subsection}{\gs\large}{\thesubsection}{1em}{#1}

\renewcommand{\jsParagraphMark}{}

\usepackage{froufrou}

%%% 章末問題
\newcommand{\exercises}{
  \pagestyle{nc-s}
  \sectionmark{章末問題}
  \phantomsection
  \section*{章末問題}
  \addcontentsline{toc}{section}{章末問題}
}

% 
% 
% 

%%%%% ウォーターマーク/ラベル表示
\myboolT{BoolUseDraft}{
  \usepackage{draftwatermark}
  \usepackage[notref]{showkeys}
  \renewcommand*{\showkeyslabelformat}[1]{\normalfont\scriptsize\ttfamily\llap{\fbox{#1}}}
}

% 
% 
% 

%%%%% ハイパーリンク
\usepackage{hyperref}
\hypersetup{setpagesize=false, bookmarksdepth=4, bookmarksnumbered=true, colorlinks=true, linkcolor=mylinkcolor, citecolor=mycitecolor, urlcolor=myurlcolor, pdfauthor={鴎海}}
\makeatletter
\NewDocumentCommand{\mylabel}{mO{}}{
  \bgroup
    \protected@edef\@currentlabelname{#2}
    \label{#1}
  \egroup
}
\makeatother
\newcommand{\phantomlabel}[1]{\phantomsection\label{#1}}
\newcommand{\phyperref}[2]{\hyperref[#1]{#2\textsuperscript{→p.\,\pageref*{#1}}}}

% 
% 
% 

%%%%% cref
\usepackage[nameinlink]{cleveref}
\crefname{page}{p.}{pp.}
\makeatletter
\crefformat{chapter}{#2\@chapapp #1\@chappos #3}
\makeatother
\crefformat{section}{#2\S #1#3}
\crefformat{subsection}{#2\S\S #1#3}
\crefname{equation}{式}{式}
\creflabelformat{equation}{#2(#1)#3}
% \crefname{figure}{図}{図}
% \crefname{table}{表}{表}

% 
% 
% 

%%%%% zref
\usepackage{zref-xr}
\zxrsetup{toltxlabel}

% 
% 
% 

%%%%% リスト
\usepackage{tasks}
\usepackage[inline]{enumitem}
\setlist{align=left, leftmargin=*}
% - - - - - - - - - -
\newlist{myenum}{enumerate}{3}
\setlist[myenum, 1]{label=(\roman{myenumi}), ref=(\roman{myenumi})}
\setlist[myenum, 2]{label=(\roman{myenumi}.\roman{myenumii}), ref=(\roman{myenumi}.\roman{myenumii})}
\setlist[myenum, 3]{label=(\roman{myenumi}.\roman{myenumii}.\roman{myenumiii}), ref=(\roman{myenumi}.\roman{myenumii}.\roman{myenumiii})}
\newlist{myenum*}{enumerate*}{1}
\setlist[myenum*]{label=(\roman*), ref=(\roman*)}
% - - - - - - - - - -
\newlist{thmlist}{enumerate}{2}
\setlist[thmlist, 1]{label=\arabic{thmlisti}., ref=\thetcbcounter.\arabic{thmlisti}}
\setlist[thmlist, 2]{label=\arabic{thmlisti}.\arabic{thmlistii}., ref=\thetcbcounter.\arabic{thmlisti}.\arabic{thmlistii}}
% - - - - - - - - - -
\newlist{proc}{enumerate}{3}
\setlist[proc, 1]{label=(\arabic{proci}), ref=(\arabic{proci})}
\setlist[proc, 2]{label=(\arabic{proci}.\arabic{procii}), ref=(\arabic{proci}.\arabic{procii})}
\setlist[proc, 3]{label=(\arabic{proci}.\arabic{procii}.\arabic{prociii}), ref=(\arabic{proci}.\arabic{procii}.\arabic{prociii})}
% - - - - - - - - - -
\newlist{exc}{enumerate}{1}
\setlist[exc]{label=問\arabic*., ref=\arabic*}
\crefname{exci}{問}{問}
% - - - - - - - - - -
% descriptionのitemへのリンク
% \makeatletter
% \NewDocumentCommand{\itemlabel}{mmO{}}{
%   \item[#1]
%   \phantomsection
%   \ifstrempty{#3}{\renewcommand{\@currentlabel}{#1}}{\renewcommand{\@currentlabel}{#3}}
%   \label{#2}
% }
% \makeatother

% 
% 
% 

%%%%% 数式
\everymath=\expandafter{\the\everymath \narrowbaselines\displaystyle}
\usepackage{scalerel}
\allowdisplaybreaks[4]

% 
% 
% 

%%%%% アルゴリズム
\usepackage[showColor=mygreen, asmpColor=myred]{../formalproofwriter}

% 
% 
% 

%%%%% tcolorbox
\usepackage{tcolorbox}
\tcbuselibrary{skins, breakable}
\NewTColorBox[auto counter, number within=chapter]{none}{}{}
% 1: 略称. 2: 日本語名. 3: 色. 4: 下側コンテンツ
\NewDocumentCommand{\defitem}{mmmO{}}{%
  \crefname{thmlist#1}{#2}{#2}
  % 1: label. 2: nameref. 3: subtitle.
  \NewTColorBox[use counter from=none, crefname={#2}{#2}]{#1}{O{}O{}D""{}}{%
    enhanced, breakable,
    boxrule=0pt, frame hidden,
    sharp corners=west,
    colback=#3, opacityback=0.1, skin=bicolor, opacitybacklower=0,
    coltitle=#3, fonttitle=\sffamily,
    borderline west={2pt}{0pt}{#3},
    attach boxed title to top left={xshift=-12pt},
    boxed title style={tile, size=minimal, opacityback=0, bottom=2pt, borderline south={.5pt}{0pt}{#3}},
    IfBlankF={##1}{label={##1}},
    nameref={##2},
    title={#2\thetcbcounter\IfBlankF{##3}{:##3}},
    before upper={\crefalias{thmlisti}{thmlist#1}},
    before lower={\IfBlankF{#4}{\textsf{#4}\hspace{1\zw}}},%
  }%
}
\newcommand{\tcbex}{{\gsb 例:}\hspace{1em}}
\defitem{rgl}{規約}{myblue}[例]
\defitem{dfn}{定義}{myblue}[例]
\defitem{mdfn}{メタ定義}{myblue}[例]
\defitem{fct}{事実}{mygreen}[論証]
\defitem{lem}{補題}{mygreen}[証明]
\defitem{prp}{命題}{mygreen}[証明]
\defitem{mthm}{メタ定理}{mygreen}[証明]
\defitem{thm}{定理}{mygreen}[証明]
\defitem{asm}{前提}{myred}[例]
\defitem{axm}{公理}{myred}[例]
\defitem{cnv}{約束}{mygray}[例]

% 注意
% \NewTColorBox{nb}{O{}}{%
%   enhanced,
%   breakable,
%   colframe=black, colback=white, coltitle=black, colbacktitle=white,
%   borderline={.7pt}{0pt}{dashed},
%   frame hidden,
%   attach boxed title to top text left={yshift*=-\tcboxedtitleheight/2},
%   boxed title style={tile}, boxed title size=title,
%   title={\gs 注意\IfBlankF{#1}{(#1)}},
% }

% 
% 
% 

%%%%% 表など
\usepackage{xltabular}
\usepackage{booktabs}

% 
% 
% 

%%%%% TikZ
% \usetikzlibrary{}

% 
% 
% 

%%%%% 索引
\ExplSyntaxOn
\keys_define:nn { idx } {
  child .tl_set:N = \l_idx_child_tl,
  child-sort .tl_set:N = \l_idx_childsort_tl,
  entry .tl_set:N = \l_idx_entry_tl,
  entry-sort .tl_set:N = \l_idx_entrysort_tl
}
\NewDocumentCommand { \idx } { mmO{}O{} } {
  \group_begin:
  \keys_set:nn { idx } { #4 }
  \textsf{#1}
  \tl_if_empty:nTF { #3 } {} { (#3) }
  \tl_if_empty:NTF \l_idx_child_tl {
    \tl_if_empty:NTF \l_idx_entry_tl {
      \index{#2@#1}
    } {
      \index{\l_idx_entrysort_tl@l_idx_entry_tl}
    }
  } {
    \tl_if_empty:NTF \l_idx_entry_tl {
      \index{#2@#1!\l_idx_childsort_tl@\l_idx_child_tl}
    } {
      \index{\l_idx_entrysort_tl@l_idx_entry_tl!\l_idx_childsort_tl@\l_idx_child_tl}
    }
  }
  \group_end:
}
\ExplSyntaxOff
\myboolTF{BoolShowIndex}{
  \usepackage{makeidx}
  \makeindex
}{
  \newcommand{\printindex}{}
  \RenewDocumentCommand{\idx}{mmO{}O{}}{\textsf{#1}\IfBlankF{#3}{(#3)}}
}

% 
% 
% 

%%%%% 参考文献
\myboolTF{BoolShowBib}{
  \usepackage[style=authoryear, backend=biber]{biblatex}
  \addbibresource{biblio.bib}
  \DeclareNameAlias{default}{family-given}
  \newcommand{\myprintbibliography}{
    \nocite{*}
    \printbibliography[title=参考文献, label=biblio]
  }
}{
  \newcommand{\myprintbibliography}{}
}

% 
% 
% 

%%%%% グロッサリー
\myboolTF{BoolShowGls}{
  \usepackage[record]{glossaries-extra}
  \GlsXtrLoadResources[src=symbols, set-widest, sort=use]
  \newcommand{\myprintglossary}{\printunsrtglossary[style=symbol-list]}
  \newglossarystyle{symbol-list}{
    \renewenvironment{theglossary}{
      \begin{longtable}[l]{@{}cll@{}}
    }{%
      \end{longtable}
    }
    \renewcommand*{\glossaryheader}{%
      \toprule
      {\gs 記号} & {\gs 説明} & {\gs 項目} \\
      \midrule
      \endfirsthead
      \toprule
      {\gs 記号} & {\gs 説明} & {\gs 項目} \\
      \midrule
      \endhead
      \bottomrule
      \endfoot
      \bottomrule
      \endlastfoot
    }
    \renewcommand*{\glossentry}[2]{%
      \glossentrysymbol{##1} & \glossentrydesc{##1} & \glstarget{##1}{\glossentryname{##1}} \tabularnewline
    }
    \renewcommand*{\glossarytitle}{記号一覧}
    \renewcommand*{\glossarytoctitle}{記号一覧}
  }
}{
  \newcommand{\glsadd}[1]{}
  \newcommand{\myprintglossary}{}
}

% 
% 
% 

%%%%% 文書テンプレート
\NewDocumentCommand{\Title}{mO{鴎海}}{
  \title{\gs\Large 学部数学のための\\ \gsb\huge #1}
  \author{{\gs 編:}#2}
  \date{\small (最終更新日:\today)}
}
\newcommand{\BeforeMainContent}{
  % タイトル
  \pagenumbering{roman}

  \maketitle

  \pagestyle{c-c}

  % イントロダクション
  \myboolT{BoolShowIntro}{\input{intro.tex}}

  % 目次
  \tableofcontents

  % 本文開始
  \cleardoublepage
  \pagenumbering{arabic}
  \pagestyle{nc-ns}
}
\newcommand{\AfterMainContent}{
  \cleardoublepage
  \pagestyle{c-c}

  % 参考文献
  \myprintbibliography

  % 索引
  \cleardoublepage
  \printindex

  % 記号索引
  \myprintglossary
}
